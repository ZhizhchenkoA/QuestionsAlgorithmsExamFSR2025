\section{Потоки в сетях. Задача о максимальном потоке. Алгоритм Форда – Фалкерсона.}
Введём необходимые определения
\begin{definition}
\textbf{Сеть} $G = (V, E)$  --- ориентированный граф, в котором:
\begin{enumerate}
	\item каждое ребро $(u,v) \in E$
	имеет положительную пропускную способность $c(u, v) > 0$ (англ. capacity). Если $(u, v) \notin E$, то $c(u, v) = 0$,
	\item выделены две вершины --- \textbf{исток} $s$ и \textbf{сток} $s$.
\end{enumerate}
\end{definition}

\begin{definition}
	\textbf{Поток} (англ. flow) $f$ в $G$ --- действительная функция $f: V \times V \to \mathbb{R}$, удовлетворяющая условиям:
	\begin{enumerate}
		\item $f(u, v) = -f(v, u)$ (антисимметричность);
		\item $f(u, v) \le c(u, v)$ (ограничение пропускной способности), если ребра нет, то $f(u, v) = 0$;
		\item $\sum_v f(u, v) = 0$ для всех вершин $u$, кроме $s$ и $t$ (закон сохранения потока).
	\end{enumerate}
\end{definition}

\begin{definition}
	 \textbf{Величина потока} $f$ определяется как $$\displaystyle |f| = \sum_{v \in V} f(s, v) = \sum_{v \in V} f(v,t)$$.
\end{definition}


В задаче о \textbf{нахождении максимального потока} дана некоторая сеть $G$ с истоком $s$ и стоком $t$ и требуется найти поток максимальной величины.

Дальнейшие определения нам понадобятся для формулировки самого алгоритма
\begin{definition}
	\textbf{Остаточной пропускной способностью} (англ. residual capacity) ребра $(u,v)$ называется величина дополнительного потока, который мы можем направить из $u$ в $v$, не превысив пропускную способность $c(u,v)$. Иными словами
	$$c_f(u,v) = c(u,v) - f(u,v).$$
\end{definition}

\begin{definition}
	Для заданной транспортной сети $G=(V,E)$ и потока $f$, \textbf{остаточной сетью} (дополняющая сеть, англ. residual network) в $G$, порожденной потоком $f$, является сеть $G_f=(V,E_f)$, где $E_f=\{(u,v) \in V \times V \mid c_f(u,v) > 0\}$.
\end{definition}

\begin{definition}
	Для заданной транспортной сети $G=(V,E)$ и потока $f$ \textbf{дополняющим путем} (англ. augmenting path) $p$ является простой путь из \textbf{истока} в \textbf{сток} в остаточной сети $G_f=(V,E_f)$.
\end{definition}

\begin{definition}
	
Определения, которые понадобятся для доказательства корректности алгоритма
\textbf{$(s,t)$-разрезом} (англ. $s$-$t$ cut) $\langle S,T \rangle$ в сети $G$ называется пара множеств $S, T$, удовлетворяющих условиям:
\begin{enumerate}
	\item $s \in S, t \in T$
	\item $S = V \setminus T$
\end{enumerate}
\end{definition}

\begin{definition}
\textbf{Пропускная способность разреза} (англ. capacity of the cut) $\langle S,T \rangle$ обозначается $c(S,T)$ и вычисляется по формуле:
$$c(S,T) = \sum_{u \in S} \sum_{v \in T} c(u, v).$$
\end{definition}

\begin{definition}
	\textbf{Поток в разрезе }(англ. flow in the cut) $\langle S,T \rangle$ обозначается $f(S,T)$ и вычисляется по формуле:
	$$f(S,T) = \sum_{u \in S} \sum_{v \in T} f(u, v).$$
\end{definition}
 


\begin{definition}
	\textbf{Минимальным разрезом} (англ. minimum cut) называется разрез с минимально возможной пропускной способностью.
	
\end{definition}

Сначала найдём связь между максимальным потоком и потоком через разрез

\begin{theorem}{(Форд --- Фалкерсон)}\\
	Поток через минимальный разрез равен максимальному потоку.
\end{theorem}

\begin{proof}
	Сумма потоков из $s$ равна потоку через любой разрез, в том числе минимальный, следовательно, не превышает пропускной способности минимального разреза. Следовательно, максимальный поток не больше пропускной способности минимального разреза.
	
	Осталось доказать, что он и не меньше её. Пускай поток максимален. 
	Тогда в остаточной сети сток не достижим из источника. 
	Пусть $A$ --- множество вершин, достижимых из источника в остаточной сети, $B$ --- недостижимых. Тогда, поскольку $s \in A, t \in B$, то $(A,B)$ является разрезом. 
	Кроме того, в остаточной сети не существует ребра $(a, b)$ с положительной пропускной способностью, такого что $a \in A, b \in B$, иначе $b$ было бы достижимо из $s$. 
	Следовательно, в исходной сети поток по любому такому ребру равен его пропускной способности, и, значит, поток через разрез $(A,B)$ равен его пропускной способности. 
	Но поток через любой разрез равен суммарному потоку из источника, который в данном случае равен максимальному потоку. 
	Значит, максимальный поток равен пропускной способности разреза $(A,B)$, которая не меньше пропускной способности минимального разреза. 
	(ч. т. д)
\end{proof}

\subsection*{Сам алгоритм}
\begin{enumerate}
	\item Обнуляем все потоки. Остаточная сеть изначально совпадает с исходной сетью.
	\item В остаточной сети находим любой путь из источника в сток. Если такого пути нет, останавливаемся.
	\item Пускаем через найденный путь (он называется увеличивающим путём или увеличивающей цепью) максимально возможный поток:
	\begin{enumerate}
		\item На найденном пути в остаточной сети ищем ребро с минимальной пропускной способностью $c_{\min}$.
		\item Для каждого ребра на найденном пути увеличиваем поток на $c_{\min}$, а в противоположном ему - уменьшаем на $c_{\min}$.
		\item Модифицируем остаточную сеть. Для всех рёбер на найденном пути, а также для противоположных им рёбер, вычисляем новую пропускную способность. Если она стала ненулевой, добавляем ребро к остаточной сети, а если обнулилась, стираем его.
	\end{enumerate}
	\item Возвращаемся на шаг 2.
\end{enumerate}

