\section{Диаграммы Вороного. Алгоритм Форчуна. }

\begin{definition}
	Обозначим  $dist(р, q)$  евклидово  расстояние  между  двумя  точками $p$ и $q$.
	На  плоскости имеет место формула: $$dist(p, q) = \sqrt{(p_x-q_x)^2+(p_y-q_y)^2}.$$
	Пусть  $P  := \{p_1,  p_2, ..., p_n\}$  ---  множество  $n$  различных  точек  на  плоскости.
	Определим  диаграмму  Вороного  множества  $P$  как разбиение  плоскости  на  $n$  ячеек,  по  одной  для  каждого  центра  из  $P$,  обладающее тем  свойством,  что  точка  $q$  принадлежит  ячейке,  соответствующей  центру  $p_i$,  тогда и  только  тогда,  когда  $distq, p_i)  <  dist(q,  p_j)$  для  любой  точки  $p \in P$ с $i \not= j$.
	Диаграмму Вороного  множества  $P$  будем  обозначать  $Vor(P)$. 
	Допуская  некоторую  вольность выражений,  мы  иногда  будем  употреблять  термин  «$Vor(P)$»  или  «диаграмма  Вороного»  для  обозначения  одних  лишь  ребер  и  вершин  разбиения.
	Например,  говоря, что  диаграмма  Вороного  связна,  мы  имеем  в  виду,  что  объединение  ребер  и  вершин образует  связное  множество. 
	Ячейка  $Vor(P)$,  соответствующая  центру  $p_i$,  обозначается  $\cal V(p_i)$;  будем  называть  ее  ячейкой  Вороного  для  $p_i$.
\end{definition}

Теперь  присмотримся  к  диаграмме  Вороного  повнимательнее. 
Сначала  изучим строение  одной  ячейки  Вороного. 
Для  двух  точек  $p$  и  $q$  на  плоскости  назовем  срединным  перпендикуляром  $p$  и  $q$  перпендикуляр,  к  отрезку  $pq$,  проходящий  через его  середину. 
Срединный  перпендикуляр  делит  плоскость  на  две  полуплоскости. 
Обозначим  $h(p, q)$  открытую  полуплоскость,  содержащую  $p$,  a  $h(q, p$  -  открытую полуплоскость,  содержащую  $q$. 
Заметим,  что  $r \in h(p, q)$  тогда  и  только  тогда,  когда $dist(r,p)  <  dist(r,  q)$. 
Отсюда  вытекает  следующее  наблюдение.

\begin{observation}
	$V(p_i) = \cap_{1 \le j \le n, i \not= j} h(p_i, p_j)$.
\end{observation}

Таким  образом,  $V(p_i)$  ---  пересечение  $n  -  1$  полуплоскостей, а,  значит,  является  выпуклой  многоугольной  областью  (возможно,  неограниченной),  имеющей  не  более $n - 1$ вершин  и не  более $n - 1$  ребер.

Как  выглядит  полная  диаграмма  Вороного? 
Мы  только что  видели,  что  каждая  ячейка  диаграммы  представляет  собой  пересечение  нескольких  полуплоскостей,  поэтому  диаграмма  Вороного  ---  это планарное  разбиение  с  прямолинейными  ребрами.
Одни  ребра  являются  отрезками,  другие  ---  полупрямыми. 
Если  только  не  все  центры  коллинеарны,  то  ребер, являющихся  полными  прямыми,  не  будет.

\begin{theorem}
	Пусть $P$ --- множество, содержащее  $n$  точек  на  плоскости  (центров). 
	Если  все центры  коллинеарны,  то  диаграмма  Вороного  состоит  из $n - 1$  параллельных  прямых. 
	Иначе  $Vor(P)$ связна,  и  ее  ребра  являются  отрезками  или  полупрямыми.
\end{theorem}
\begin{proof}
	Первую  часть  теоремы  доказать  легко,  поэтому  предположим,  что  не  все  центры коллинеарны.
	Сначала  покажем,  что  ребра  $Vor(P)$  являются  либо  отрезками,  либо  полупрямыми. 
	Мы  уже  знаем,  что  ребра  $Vor(P)$  ---  части  прямых  линий,  а  именно  срединных перпендикуляров  пары  центров. 
	Предположим  противное  ---  что  ребро $e$ диаграммы $Vor(P)$ является  полной  прямой. 
	Пусть  $e$  лежит  на  границе ячеек  Вороного $V(p_i)$ и $V(p_j)$.
	Пусть $p_k \in P$  ---  точка,  не  лежащая на  одной  прямой  с  $p_i$ и $p_j$.
	Срединный  перпендикуляр  $p_j$  и  $p_k$  не параллелен  $e$,  а,  значит,  пересекает  $e$. 
	Но  тогда  часть  $e$,  лежащая внутри  $h(p_k,  p_j)$,  не  может  находиться  на  границе  $V(p_j)$,  поскольку  она  ближе  к  $p_k$,  чем  к  $p_j$.
	Мы  получили  противоречие.
	
	Остается  доказать,  что  $Vor(P)$  связна. 
	Если  бы  это  было  не так,  то  существовала  бы  ячейка  Вороного  $V(p_i)$,  делящая  диаграмму  на  две  части. 
	Поскольку  ячейки  Вороного  выпуклы,  то  $V(p_i)$  состояла  бы  из  полосы,  ограниченной  двумя  параллельными  прямыми.
	Но  мы  только  что  доказали,  что  ребра  диаграммы  Вороного  не  могут  быть  полными  прямыми. 
	Противоречие!
\end{proof}

Теперь,  понимая  строение  диаграммы  Вороного,  исследуем  ее  сложность,  т.  е.  оценим  общее  число  вершин  и  ребер. 
Поскольку  существует  $n$  центров  и  у  каждой чейки  Вороного  не  более  $n - 1$  вершин  и  ребер,  то  сложность $Vor(P)$  в  худшем случае  квадратичная. 
Не  ясно,  однако,  действительно  ли  $Vor(P)$  может  иметь  квадратичную  сложность:  легко  построить  пример,  когда  сложность  ячейки  Вороного линейна,  но  может  ли  оказаться,  что  у  многих  ячеек  линейная  сложность? 
Следующая  теорема  показывает,  что  это  не  так  и  что  среднее  число  вершин  ячейки Вороного  меньше  шести.

\begin{theorem}
	Для  $n > 3$  число  вершин  в  диаграмме  Вороного  множества  $n$  точек  на плоскости  не  превосходит  $2 n - 5$,  а  число  ребер  не  больше  $3 n - 6$.
\end{theorem}
\begin{proof}
	Если  все  центры  коллинеарны,  то теорема  сразу  следует  из  предыдущей,  поэтому  будем предполагать,  что  это  не  так.
	Используем  для  доказательства  формулу  Эйлера,  согласно  которой  для  любого связного  планарного  графа  с $m_v$  вершинами,  $m_e$  ребрами и  $m_f$  гранями  имеет  место  следующее  соотношение: $$m_v - m_e + m_f = 2.$$
	
	Мы  не  можем  применить  формулу  Эйлера  непосредственно  к  $Vor(P)$,  потому  что  $Vor(P)$  содержит  полубесконечные  ребра  и  потому  не  является  настоящим  графом.
	Чтобы  исправить  положение,  добавим  одну  дополнительную  вершину $v_\infty$ находящуюся  «в  бесконечности»,  и соединим  все  полубесконечные  ребра  $Vor(P)$  с  этой  вершиной. 
	Теперь  мы  получили  связный  планарный  граф,  к которому  можно  применить  формулу  Эйлера. 
	Получаем следующее  соотношение  между  $n_v$,  количеством  вершин $Vor(P)$,  $n_e$,  количеством  ребер  $Vor(P)$,  и  $n$,  количеством граней: 
	\begin{equation}
		\label{eq:13.1}
		(n_v + 1) - n_e + n = 2.
	\end{equation}
	
	Далее,  каждое  ребро  в  пополненном  графе  имеет  ровно  две  вершины,  поэтому если  просуммировать  степени  всех  вершин,  то  получится  удвоенное  количество рёбер.
	Поскольку  степень  каждой  вершины,  включая  $v_\infty$,  не  меньше  трех,  получаем:
	\begin{equation}
		\label{eq:13.2}
		2 n_e \ge 3 (n_v + 1)
	\end{equation}
	
	В  сочетании  с  формулой  (\ref{eq:13.1})  это  доказывает  теорему.
\end{proof}

Мы  знаем,  что  ребра  являются частями  срединных  перпендикуляров  пар  центров  и  что  вершины  ---  это  точки  пересечения  срединных  перпендикуляров. 
Количество  срединных  перпендикуляров  квадратично  зависит  от числа  центров,  тогда  как  сложность  $Vor(P)$  всего  лишь  линейна.
Следовательно,  не  все  срединные  перпендикуляры  определяют  ребра  $Vor(P)$,  и  не все  их  пересечения  являются  вершинами  $Vor(P)$.
Чтобы  понять,  какие  срединные перпендикуляры  и  точки  их  пересечения  формируют  отличительные  характеристики  диаграммы  Вороного,  дадим  следующее  определение.
Для  точки  $q$  определим  наибольший  пустой  круг  $q$  относительно  $P$  $(C_P(q))$  как  наибольший  круг  с центром  в  $q$,  который  не  содержит  внутри  себя  ни  одного  центра  из  $P$. 
Следующая теорема  характеризует  вершины  и  ребра  диаграммы  Вороного.

\begin{theorem}
	Для  диаграммы  Вороного  $Vor(P)$  множества  точек  $P$  справедливы следующие  утверждения:
	
	\begin{enumerate}
		\item Точка  $q$  является  вершиной  $Vor(P)$  тогда  и  только  тогдау  когда  граница  ее наибольшего  пустого  круга  $C_P(q)$  содержит  три  или  более  центров  из  $P$.
		\item Срединный  перпендикуляр  центров  $p_i$  и  $p_j$  определяет  ребро  $Vor(P)$  тогда  и только  тогда,  когда  существует  точка  $q$  на  нем  такая,  что  граница  $C_P(q)$ содержит  оба  центра  $p_i$  и  $p_j$  и  никаких  других  центров.
	\end{enumerate}
\end{theorem}

\begin{proof}
	\begin{enumerate}
		\item Предположим,  что  существует  такая  точка  $q$,  что  граница $C_P(q)$  содержит  три  или  более  центров.
		Пусть  $p_i$,  $p_j$  и  $p_k$ ---  три  таких  центра.
		Поскольку  внутренность  $C_P(q)$  пуста,  $q$  должна  лежать  на  границе  каждой  из  ячеек  $V(p_i)$,  $V(p_j)$  и  $V(p_k)$,  и  $q$  должна  быть  вершиной  $Vor(P)$.
		
		С  другой  стороны,  каждая  вершина  $q$  диаграммы $Vor(P)$  инцидентна  по  меньшей  мере  трем  ребрам  и,  следовательно,  по  меньшей  мере  трем  ячейкам  Вороного $V(p_i)$,  $V(p_j)$  и  $V(p_k)$. 
		Вершина  $q$  должна  находиться  на  равных  расстояниях  от  $p_i$,  $p_j$  и  $p_k$,  и  не  может  существовать другого  центра,  более  близкого  к  $q$,  поскольку  в  противном  случае  $V(p_i)$,  $V(p_j)$  и  $V(p_k)$  не  сошлись  бы  в  $q$.
		Следовательно,  внутренность  круга,  на  границе  которого  лежат $p_i$,  $p_j$  и  $p_k$,  не  содержит  ни  одного  центра.
		
		\item Предположим,  что  существует  точка  $q$,  обладающая  указанным  в  теореме свойством.
		Поскольку  внутренность  круга  $C_P(q)$  не  содержит  ни  одного  центра,  а центры  $p_i$  и  $p_j$  лежат  на  его  границе,  то  $dist(q, p_i)  =  dist(q, p_j) \le dist(q,  p_k)$  для  любого $1 \le k \le n$.
		Отсюда  следует,  что  $q$  лежит  на  ребре  или  является  вершиной  $Vor(P)$.
		Но из  первой  части  теоремы  вытекает,  что  $q$  не  может  быть  вершиной  $Vor(P)$.
		Значит,  $q$ лежит  на  ребре  $Vor(P)$,  которое  определяется  срединным  перпендикуляром $p_i$ и $p_j$.
		
		Обратно,  пусть  срединный  перпендикуляр  $p_i$ и $p_j$ определяет  ребро  диаграммы Вороного.
		На  границе  наибольшего  пустого  круга  любой  точки  $q$  во  внутренней части  этого  ребра  должны  находиться $p_i$ и $p_j$ и  ни  одного  другого  центра.
	\end{enumerate}
\end{proof}

\subsection{Алгоритм Форчуна}

Принятая  в  алгоритме  стратегия  заключается  в  заметании  плоскости  прямой, опускающейся  сверху  вниз.
В  процессе  заметания  мы  пересчитываем  показатели, описывающие  вычисляемую  структуру.
Точнее,  пересчитывается  информация  о пересечении  структуры  с  заметающей  прямой.
Эта  информация  изменяется  лишь в  некоторых  точках --- \textit{точках  событий}.

Попробуем  применить  эту  общую  стратегию  к  вычислению  диаграммы  Вороного  множества $P = \{p_1, p_2, ..., p_n\}$ центров  на  плоскости.
В  соответствии  с  идеей  заметания плоскости  мы  двигаем  горизонтальную  заметающую  прямую  $l$  сверху  вниз.
При  этом  мы  должны  пересчитывать пересечение  диаграммы  Вороного  с  заметающей  прямой.
К  сожалению,  это  не  так  просто,  потому  что  часть  $Vor(P)$ над $l$ зависит  не  только  от  центров,  лежащих  выше  $l$,  но  и от  центров,  лежащих  ниже $l$.
Иначе  говоря,  когда  заметающая  прямая  достигает  самой  верхней  вершины  ячейки Вороного $V(p_i)$,  она  еще  не  встретила  соответствующий  центр $p_i$.
Следовательно, у  нас  нет  информации,  необходимой  для  вычисления  вершины.
Мы  вынуждены применить  идею  заметания  плоскости  немного  по-другому:  вместо  пересчета  пересечения  диаграммы  Вороного  с заметающей  прямой,  будем  хранить  информацию о  части  диаграммы  Вороного  для  центров  выше $l$,  которая  не  может измениться из-за  центров  ниже $l$.

Обозначим $l^+$ замкнутую  полуплоскость,  расположенную  выше  $l$.
Какая  часть диаграммы  Вороного  выше $l$ больше  не  может  измениться? 
Иначе  говоря,  для  каких  точек $q \in l^+$ мы  точно  знаем  ближайший  к  ним  центр?
Расстояние  от  точки $q \in l^+$  до  любого  центра,  находящегося  под  $l$,  больше  расстояния  от $q$ до  самой $l$.  Поэтому  ближайший  к  $q$  центр  не  может  находиться  ниже $l$,  если  $q$  удалена  от какого-то  центра $p_i \in l^+$ на  расстояние,  не  большее,  чем  $q$  от $l$.
Геометрическое  место  точек,  расположенных  к  некоторому  центру  $p_i \in l^+$,  ближе,  чем  к $l$,  ограничено параболой.
Поэтому  геометрическое  место  точек,  расположенных  ближе  к  какому-нибудь  центру,  находящемуся  выше $l$,  чем  к  самой  $l$,  ограничено  дугами  парабол.
Назовем  эту  последовательность  параболических  дуг  \textit{линией  прибоя}.
Линию прибоя  можно  наглядно  представить  и  по-другому.  Каждый  центр $p_i$,  находящийся выше  заметающей  прямой,  определяет  полную  параболу $\beta_i$.
Линия  прибоя --- это график  функции,  которая  для  каждой  координаты $x$ принимает  значение,  совпадающее  с  самой  нижней  точкой  всех  таких  парабол.

\begin{observation}
	Линия  прибоя  $x$-монотонна,  т.  е. каждая  вертикальная  прямая  пересекает  ее  ровно  в  одной точке.
\end{observation}

Легко  видеть,  что  одна  парабола  может  несколько раз  встречаться  в  линии  прибоя. 
О  том,  сколько  именно раз,  мы  подумаем  позже. 
А  пока  заметим,  что  точки  излома  на  стыке  дуг  разных  парабол,  образующих  линию прибоя,  лежат  на  ребрах  диаграммы  Вороного.
И  это  не случайное  совпадение:  точки  излома  точно  вычерчивают  диаграмму  Вороного  по  мере  того,  как  заметающая  прямая  опускается  вниз. 
Эти  свойства  линии  прибоя  легко  доказать  с  помощью  элементарных  геометрических  рассуждений.

Поэтому  вместо  того  чтобы  пересчитывать  пересечение  $Vor(P)$  с $l$ по  мере  опускания  заметающей  прямой,  мы  будем  пересчитывать  линию  прибоя.
Мы  не  станем  хранить  линию  прибоя  явно,  потому  что  она  непрерывно  изменяется  при  перемещении $l$.
Ненадолго  оставим  в  стороне  вопрос  о  том,  как  представлять  линию прибоя,  и  сначала  разберемся,  когда  и  как  изменяется  ее  комбинаторная  структура.
Это  происходит,  когда  появляется  новая  параболическая  дуга и  когда  старая дуга  схлопывается  в  точку  и  исчезает.

Сначала  рассмотрим  событие  появления  новой  дуги  на  линии  прибоя.
Во-первых,  это  может  случиться,  когда  заметающая  прямая $l$  доходит  до  нового  центра.
В  первый  момент  парабола,  определяемая  этим  центром,  вырождена  и  имеет нулевую  ширину:  это  вертикальный  отрезок,  соединяющий  новый  центр  с  линией  прибоя.
По  мере  того  как  заметающая  прямая  продолжает  опускаться,  новая парабола  расширяется.
Часть  новой  параболы  ниже  старой  линии  прибоя  теперь становится  частью  новой  линии  прибоя.
Будем называть  встречу  заметающей  прямой  с  новым  центром  \textit{событием  центра}.

Что  происходит  с  диаграммой  Вороного  в  момент  события  центра?
Напомним, что  точки  излома  на  линии  прибоя  вычерчивают  ребра диаграммы  Вороного.
В  момент  события  центра  появляются  две  новые  точки  излома,  которые  начинают  вычерчивать  ребра. 
В  начальный момент  эти  точки  излома  совпадают,  а  затем  расходятся в  разные  стороны,  вычерчивая  одно  и  то  же  ребро.
Сначала  это  ребро  не  связано  с  частью  диаграммы  Вороного над  заметающей  прямой.
Но  впоследствии  ---  скоро  мы увидим,  когда  именно,  ---  растущее  ребро  наталкивается на  другое  ребро  и  соединяется  с  остальной  частью  диаграммы.

Итак,  мы  теперь  понимаем,  что  происходит  в  момент  события  центра:  на  линии  прибоя  появляется  новая  дуга  и  начинается  вычерчивание  нового  ребра  диаграммы  Вороного. 
Может  ли  новая  дуга  появиться  на  линии  прибоя  еще  каким-то способом? 
Нет,  не  может.

\begin{lemma}
	Единственная  причина  появления  новой  дуги  на  линии  прибоя  ---  событие  центра.
\end{lemma}
\begin{proof}
	Предположим  противное  ---  что  уже существующая  парабола  $\beta_j$,  определенная  центром  $p_j$, «втискивается»  в линию  прибоя. 
	Это  могло  бы  произойти  двумя  способами.
	
	Первая  возможность  --- $\beta_j$  втискивается  в  середине дуги  какой-то  параболы $\beta_i$.
	В  тот  момент,  когда  это  происходит, $\beta_i$  и  $\beta_j$  касаются,  т.  е.  имеют  ровно  одну  общую  точку.
	Обозначим  $l_y$  координату  $y$  заметающей  прямой  в  момент  касания.
	Если $p_j = (p_{j,x}, p_{j,y})$,  то  парабола $\beta_j$ описывается  уравнением $$y = \frac{1}{2 (p_{j,y} - l_y)} (x^2 - 2 p_{j,x} x + p_{j,x}^2 + p_{j,y}^2 - l_y^2).$$
	
	Уравнение  для $\beta_i$,  разумеется,  аналогично.
	Пользуясь тем  фактом,  что $p_{j,y}$ и $p_{i,y}$ больше $l_y$,  легко  показать,  что $\beta_i$ и $\beta_j$ не  могут  иметь  только  одну  точку  пересечения.
	Поэтому  парабола $\beta_j$  не  может  втиснуться  в  середину  дуги другой  параболы $\beta_i$.
	
	Вторая  возможность  --- $\beta_j$  втискивается  между  двумя  дугами.
	Пусть  эти  дуги  являются  частями  парабол $\beta_i$  и  $\beta_k$.  Обозначим  $q$  точку  пересечения  $\beta_i$ и  $\beta_k$,  в  которой $\beta_j$ только-только  появляется  на  линии  прибоя,  и  предположим,  что  $\beta_i$  расположена на  линии  прибоя  левее  $q$,  $\beta_k$  ---  правее  $q$.
	Тогда  существует  окружность  $C$,  проходящая  через  $p_i$, $p_j$  и  $p_k$ ---  центры,  определяющие  параболы.
	Эта  окружность  касается  некоторой  заметающей  прямой  $l$.
	Если  начать  с  точки  касания  и двигаться  по  часовой  стрелке,  то  точки  будут  расположены  на  окружности  $C$  в порядке $p_i$, $p_j$,  $p_k$,  поскольку  мы  предположили,  что $\beta_j$  втискивается  между  дугами  $\beta_i$  и  $\beta_k$.
	Рассмотрим  бесконечно  малое  смещение  заметающей  прямой  вниз  при  сохранении  касания  окружности  $C$  с  $l$.
	Тогда  не  может  быть  так,  что  внутри $C$  нет  ни  одного  центра,  и  при  этом  она  все-таки  проходит  через $p_j$.  либо $p_i$,  либо $p_k$ окажутся  внутри.  
	Поэтому  в  достаточно  малой  окрестности  $q$  парабола $\beta_j$  не  может стать  частью  линии  прибоя  при  опускании  заметающей  прямой,  т.  к.  либо  $p_i$,  либо $p_k$ окажутся  к $l$ ближе, чем $p_j$.
\end{proof}

Из  этой  леммы  сразу  следует,  что  количество  параболических  дуг  в  линии  прибоя  не  превышает $2n-1$:  каждый  встретившийся  центр  порождает  одну  новую дугу  и  вызывает  расщепление  не  более  одной  существующей  дуги  на  две,  а  никак по-другому  дуга  появиться  не  может.

Второй  тип  событий  в  алгоритме  заметания  плоскости  --- схлопывание  существующей  дуги  в  точку  и  последующее  исчезновение.
Пусть  $\alpha'$  --- исчезающая  дуга,  а  $\alpha$ и $\alpha''$ --- две  соседних  с  $\alpha'$  дуги  в  момент,  предшествующий ее  исчезновению.
Дуги  $\alpha$ и $\alpha''$  не  могут  быть  частями  одной  и  той  же  параболы; эту  возможность  можно  исключить  точно  так  же,  как  первую  возможность  в  доказательстве  последней леммы.
Поэтому  три  дуги  $\alpha$,  $\alpha'$ и $\alpha''$  определены  разными  центрами  $p_i$, $p_j$ и $p_k$.
В  момент  исчезновения  $\alpha'$  параболы,  определенные  этими  центрами,  имеют общую  точку  $q$.
Эта  точка  находится  на  равном  расстоянии  от  прямой $l$ и  каждого из  трех  центров.
Поэтому  существует  окружность,  проходящая  через  $p_i$, $p_j$ и $p_k$,  с центром  в  точке  $q$,  самая  нижняя  точка  которой  лежит  на  $l$.
Внутри  этой  окружности  не  может  находиться  ни  одного  центра  диаграммы  Вороного:  такой центр был  бы  ближе  к  $q$,  чем  $q$  к  $l$, а  это  противоречит  тому  факту,  что  $q$  расположена на  линии  прибоя.
Отсюда  следует,  что  точка  $q$ --- вершина  диаграммы  Вороного. 
Это  не  должно  вызывать  удивления,  т.  к.  выше  мы  заметили,  что  точки  излома  на линии  прибоя  вычерчивают  ребра  диаграммы  Вороного.
Таким  образом,  когда  из линии  прибоя  исчезает  дуга  и  две  точки  излома  сходятся,  должны  сойтись  и  ребра 
диаграммы.
Будем  называть  событие,  при  котором  заметающая  прямая  доходит  до самой  нижней  точки  окружности,  проходящей  через  три  центра,  которые  определяют  соседние  дуги  на  линии  прибоя,  \textit{событием  окружности}.
Из  сказанного  выше вытекает  следующая  лемма.

\begin{lemma}
	Дуга  может  исчезнуть  из  линии  прибоя  только  в  результате  события  окружности.
\end{lemma}

Итак,  мы  теперь  знаем,  когда  и  как  изменяется  комбинаторная  структура  линии  прибоя:  в  момент  события  центра  появляется  новая  дуга,  а  в  момент  события окружности  исчезает  существующая.
Мы  также  знаем,  как  это  связано  с  конструируемой  диаграммой  Вороного:  в  момент  события  центра  начинает  расти  новое ребро,  а  в  момент  события  окружности  два  растущих  ребра  встречаются  и  образуют  вершину.
Остается  подобрать  подходящие  структуры  данных  для  запоминания  необходимой  информации  в  процессе  заметания. 
Наша  цель  ---  вычислить диаграмму  Вороного,  поэтому  нужна  структура,  в  которой  будет  храниться  та часть,  которая  уже  вычислена.
Понадобятся  также  две  «стандартные»  структуры данных,  применяемые  в  любом  алгоритме  заметания  плоскости:  очередь  событий и  структура,  представляющая  состояние  заметающей  прямой.
В  данном  случае вторая  структура  будет  содержать  представление  линии  прибоя.
Ниже  описана 
реализация  всех  структур  данных.

\begin{itemize}
	\item Конструируемую  диаграмму  Вороного  мы  будем хранить  в  обычной  структуре  данных,  применяемой  для  разбиений, --- двусвязном  списке  ребер.
	Но  диаграмма  Вороного --- не  настоящее  разбиение;  некоторые  ее ребра  являются  полупрямыми  или  полными  прямыми,  а  их  представить  в  двусвязном  списке  ребер невозможно.
	В  процессе  построения  это  не  составляет  проблемы,  потому что  описанное  ниже  представление  линии  прибоя  позволит  эффективно добираться  до  нужных  частей  двусвязного  списка.
	Но  по  завершении  построения  нам  хотелось  бы  иметь  настоящий  двусвязный  список  ребер.
	Для этого  мы  добавим  большой  прямоугольник,  ограничивающий  сцену --- настолько  большой,  что  содержит  все  вершины  диаграммы  Вороного.
	Тогда окончательное  разбиение  будет  состоять  из  ограничивающего  прямоугольника  и  находящейся  внутри  него  части  диаграммы  Вороного.
	
	\item Линию  прибоя  мы  представим  сбалансированным  двоичным  деревом  поиска  Т;  это  будет  структура  состояния.
	Его  листья  соответствуют  дугам $x$-монотонной  линии  прибоя  в  порядке  следования:  самый  левый  лист представляет  самую  левую  дугу,  следующий  лист  ---  вторую  слева  дугу  и  т.  д. 
	В  каждом  листе  $\mu$  хранится  центр,  определяющий  представленную  этим  листом  дугу.
	Внутренние  узлы  Т  представляют  точки  излома  линии  прибоя. 
	Точка  излома  хранится  во  внутреннем  узле  в  виде  упорядоченного  кортежа центров $(p_i, p_j)$ где $p_i$ определяет  параболу  слева  от  точки  излома,  а $p_j$ --- справа  от  нее.
	При  таком  представлении  линии  прибоя  мы  можем  наити  дугу, расположенную  выше  нового  центра,  за  время $O(\log n)$.
	Во  внутреннем  узле мы  просто  сравниваем  координаты  $x$  нового  центра  и  точки  излома,  причем последнюю  можно  вычислить  за  постоянное  время,  зная  кортеж  центров и  позицию  заметающей  прямой.
	Отметим,  что  параболы  в  явном  виде  не хранятся.
	
	В Т  мы  храним  также  указатели  на  две  другие  структуры  данных,  используемые  в  процессе  заметания.
	В  каждом  листовом  узле  Т,  представляющем дугу  $\alpha$,  хранится  один  указатель  на  узел  в  очереди  событий,  а  именно  на тот,  что  представляет  событие  окружности,  в  момент  которого  $\alpha$  исчезает. 
	Этот  указатель  равен  NULL,  если  не  существует  такого  события  окружности, при  котором  $\alpha$  исчезает,  или  если  это  событие  еще  не  наступило.
	Наконец, в  каждом  внутреннем  узле  $v$  хранится  указатель  на  полуребро  в  двусвязном списке  ребер  диаграммы  Вороного.
	Точнее,  в  $v$  хранится  указатель  на  одно из  полуребер  того  ребра,  которое  вычерчивается  точкой  излома,  представленной $v$.
	
	\item Очередь  событий  Q  реализована  с  помощью  очереди  с  приоритетами,  где приоритетом  события  является  его  координата  $y$.
	В  очереди  хранятся  грядущие  события,  о  которых  уже  известно.
	В  случае  события  центра  мы  храним  просто  сам  центр.  
	А  для  события  окружности  храним  самую  нижнюю точку  окружности  и  указатель  на  листовый  узел  Т,  представляющий  дугу, которая  исчезнет  в  момент  возникновения  события.
\end{itemize}

Все  события  центров  известны  заранее,  события  окружностей --- нет.
И  это  подводит  нас  к  последнему  оставшемуся  вопросу  -  обнаружению  событий  окружности.

В  процессе  заметания  топологическая  структура  линии  прибоя  изменяется при  каждом  событии.
Иногда  появляются  новые  тройки  соседних  дуг,  а  иногда исчезают  существующие.
Наш  алгоритм  должен  гарантировать,  что  для  любых  трех  соседних  дуг  на  линии прибоя,  определяющих  потенциальное  событие  окружности,  это  событие  будет  помещено  в  очередь  Q.
Тут  есть две  тонкости.
Во-первых,  могут  существовать  соседние тройки,  для  которых  две  точки  излома  не  сходятся,  т.  е. они  движутся  в  таких  направлениях,  которые  исключают  встречу  в  будущем.  
Это  бывает,  когда  точки  излома движутся  вдоль  срединных  перпендикуляров,  уводящих  от  точки  пересечения.
В  таком  случае  тройка  не  определяет  потенциальное событие  окружности.
Во-вторых,  даже  если  точки  излома  тройки  сходятся, соответствующее  событие  окружности  может  не  произойти;  это  бывает,  когда  тройка исчезает  (например,  из-за  появления  нового  центра  на  линии  прибоя)  раньше,  чем 
произойдет  событие.
Такое  событие  мы  будем  называть  \textit{ложной  тревогой}.

Итак,  вот  что  делает  алгоритм.
В  момент  каждого  события  он  проверяет  все вновь  появившиеся  тройки  соседних  дуг.
Например,  в  момент  события  центра могут  образоваться  три  новые  тройки:  в  одной  новая  дуга  находится  слева,  в  другой  -  посередине,  а  в  третьей  ---  справа.
Если  у  такой  новой  тройки  точки  излома сходятся,  то  событие  помещается  в  очередь  Q.
Отметим,  что  в  случае  события  центра  тройка,  в  которой  новая  дуга  находится  посередине,  никогда  не  приводит  к событию  окружности,  потому  что  левая  и  правая  дуги  тройки  принадлежат  одной и  той  же  параболе,  а  потому  точки  излома  должны  расходиться.
Далее,  для  каждой  исчезающей  тройки  проверяется,  находится ли  соответствующее  ей  событие  в очереди  Q.
Если  да,  то  событие  является  ложной  тревогой  и  удаляется  из  очереди.
Это  легко  сделать  с  помощью  хранящихся  в  листьях  Т  указателей  на  соответствующие  события  в  очереди  Q.

\begin{lemma}
	Каждая  вершина  диаграммы  Вороного  обнаруживается  с  помощью события  окружности.
\end{lemma}
\begin{proof}
	Для  вершины  $q$  диаграммы  Вороного  пусть  $p_i$, $p_j$ и $p_k$ --- три  центра,  через  которые  проходит  окружность  $C(p_i, p_j, p_k)$,  не  содержащая  внутри  себя  ни одного  центра.
	Для  простоты  рассмотрим  только  случай,  когда  на  окружности  $C(p_i, p_j, p_k)$  нет  других  центров,  а  самая  нижняя  ее  точка  не  совпадает  ни  с  одной  из  точек $p_i$, $p_j$ и $p_k$.
	Без потери  общности  можно  предполагать,  что,  двигаясь  от  нижней  точки $C(p_i, p_j, p_k)$ по  часовой  стрелке,  мы  встретим  центры $p_i$, $p_j$ и $p_k$ именно  в  таком  порядке.
	
	Мы  должны  показать,  что  непосредственно  перед тем,  как  заметающая  прямая  достигнет  нижней  точки $C(p_i, p_j, p_k)$,  на  линии  прибоя  существуют  три  соседние дуги  $\alpha$, $\alpha'$ и $\alpha''$,  определенные  центрами $p_i$, $p_j$ и $p_k$.
	Только  в этом  случае  имеет  место  событие  окружности.  
	Рассмотрим  положение  заметающей  прямой  на  бесконечно малом  удалении  от  нижней  точки  $C(p_i, p_j, p_k)$.
	Поскольку  ни  на  самой  окружности  $C(p_i, p_j, p_k)$,  ни  внутри  нее нет  других  центров,  то  существует  окружность,  проходящая  через $p_i$ и $p_j$, которая касается  заметающей  прямой  и  не  содержит  внутри  ни  одного  центра
	Поэтому  на линии  прибоя  существуют  соседние  дуги,  определяемые  центрами  $p_i$ и $p_j$.
	Аналогично  на  линии  прибоя  существуют  соседние  дуги,  определяемые  центрами $p_j$ и $p_k$. 
	Легко  видеть,  что  обе  дуги,  определяемые  центром $p_j$, на  самом  деле  совпадают,  а отсюда  следует,  что  на  линии  прибоя  имеются  три  соседние  дуги,  определяемые центрами  $p_i$, $p_j$ и $p_k$.
	Поэтому  соответствующее  событие  окружности  находилось  в Q  непосредственно  перед  тем,  как  случиться,  а,  значит,  вершина  диаграммы  Вороного  обнаружена.
\end{proof}

Теперь  можно  описать  алгоритм  заметания  плоскости  в  деталях.
Отметим,  что после  того  как  все  события  обработаны  и  очередь  Q  опустела,  линия  прибоя  еще  не исчезла.
Остались  точки  излома,  соответствующие  полубесконечным  ребрам  диаграммы  Вороного.
Как  уже  было  сказано,  в  двусвязном  списке  ребер  нельзя  представить  полубесконечные  ребра,  поэтому  приходится  добавить  ограничивающий прямоугольник,  к  которому  эти  ребра  можно  будет  присоединить.
Ниже  приведен псевдокод  верхнего  уровня  алгоритма.

\textit{Вход.}  Множество  $P :=  \{p_1 , ...,  p_n\}$  точек  на  плоскости  (центров).

\textit{Выход.} Диаграмма  Вороного  $Vor(P)$  внутри  ограничивающего  прямоугольника, представленная  в  виде  двусвязного  списка  ребер $D$.

\begin{minted}{c}
	1. Инициализировать  очередь  событий  Q,  содержащую  все
	события  центров,  инициализировать  пустую  структуру
	состояния T и  пустой  двусвязный  список  ребер D.
	2. while  Q  не  пуста
	3.   do  извлечь  из  Q  событие  с  наибольшей  координатой  у
	4.   if  это  событие  центра  р
	5.     then  HandleSiteEvent(p)
	6.     else  HandleCircleEvent(y),  где  у  -  листовый  узел
	Т, представляющий  дугу,  которая  исчезнет
	7. Внутренние  узлы,  оставшиеся  в  T,  соответствуют
	полубесконечным  ребрам диаграммы  Вороного.  Вычислить
	ограничивающий  прямоугольник,  содержащий  внутри  себя  все
	вершины  диаграммы  Вороного,  и  присоединить
	полубесконечные  ребра  к  этому  прямоугольнику,
	соответствующим  образом  изменив двусвязный  список  ребер.
	8. Обойти  полуребра  в  двусвязном  списке,  чтобы  добавить
	записи  о  ячейках  и указатели  на  них  и  из  них.
\end{minted}

Процедуры  обработки  событий  определены  следующим  образом.

HandleSiteEvent($p_i$)

\begin{enumerate}
	\item 	Если  T  пуста,  то  вставить  в  нее  $p_i$ (так  что T будет  состоять  из  единственного листового  узла,  в  котором  хранится  $p_i$)  и  вернуться.
	В  противном  случае  выполнить  шаги  2--5.
	\item 	 Искать  в  Т  дугу  $\alpha$,  находящуюся  строго  над  $p_i$.
	Если  в  листовом  узле,  представляющем  $\alpha$,  имеется  указатель  на  событие  окружности  в  Q,  то  это  событие  является  ложной  тревогой,  и  его  следует  удалить  из  Q.
	\item 	 Заменить  листовый  узел  Т,  представляющий  $\alpha$,  поддеревом  с  тремя  листьями. 
	В  среднем  листе  хранится  новый  центр  $p_i$,  а  в  двух  других  ---  центр  $p_j$,  который раньше  хранился  вместе  с  $\alpha$  Сохранить  кортежи  $(p_i, p_j)$  и  $(p_j, p_i)$,  представляющие  новые  точки  излома,  в  двух  новых  внутренних  узлах.
	При  необходимости выполнить  перебалансировку  Т.
	\item 	 Создать  в  структуре,  описывающей  диаграмму  Вороного,  новые  записи  о  полуребрах  для  ребра,  разделяющего  $V(p_i)$  и  $V(p_j)$,  которое  будет  вычерчиваться двумя  новыми  точками  излома.
	\item 	 Проверить  тройку  соседних  дуг,  в  которой  новая  дуга,  соответствующая  $p_i$,  находится  слева,  и  определить,  сходятся  ли  точки  излома.
	Если  да,  то  вставить  в Q  новое  событие  окружности  и  добавить  указатели  между  узлом  Т  и  элементом Q.  
	Сделать  то  же  самое  для  тройки,  в  которой  новая  дуга  находится  справа.
\end{enumerate}

HandleCircleEvent($\gamma$)

\begin{enumerate}
	\item 	 Удалить  из  Т  листовый  узел  $\gamma$,  который  представляет  исчезающую  дугу  $\alpha$.
	Обновить  кортежи,  представляющие  точки  излома  во  внутренних  узлах.
	При необходимости  перебалансировать  Т.
	Удалить  из  Q  все  события  окружности, имеющие  отношение  к $\alpha$;  их  можно  найти  с  помощью  указателей  из  узла  T, предшествующего $\gamma$ и  следующего  за  ним.  
	(Событие  окружности,  в  котором $\alpha$ --- средняя  дуга,  обрабатывается  в  данный  момент  и  уже  удалено  из  Q.)
	\item 	 Добавить  центр  окружности,  породившей  событие,  в  качестве  записи  о  вершине  в  двусвязный  список  ребер  $D$),  в  котором  хранится  конструируемая  диаграмма  Вороного.
	Создать  две  записи  о  полуребрах,  соответствующие  новой  точке излома  на  линии  прибоя.
	Установить  указатели  между  ними. 
	Присоединить все  три  новые  записи  к  записям  о  полуребрах,  оканчивающихся  в  данной  вершине.
	\item 	 Проверить  новую  тройку  соседних  дуг,  в  которой  бывший  левый  сосед  а  является  средней  дугой,  и  определить,  сходятся  ли  две  точки  излома  этой  тройки. 
	Если  да,  то  вставить  в  Q  соответствующее  событие  окружности  и  установить указатели  между  новым  событием  в  Q  и  соответствующим  ему  листовым  узлом  T.
	Сделать  то  же  самое  для  тройки,  в  которой  средней  дугой  является  бывший  правый  сосед.
\end{enumerate}

\begin{lemma}
	Время  работы  описанного  алгоритма  составляет  $O(n \log n)$,  а  размер  потребляемой  памяти  равен $O(n)$.
\end{lemma}
\begin{proof}
	Примитивные  операции  над  деревом T и  очередью  событий  Q, в  частности  вставка  и  удаление  элемента,  занимают  время  $O(\log n)$.
	Примитивные операции  над  двусвязным  списком  ребер  занимают  постоянное  время.
	Для  обработки  события  мы  выполняем  постоянное  число  таких  примитивных  операций, так  что  тратим  на  этом  время $O(\log n)$.
	Очевидно,  что  всего  существует  $n$  событий центра.
	Что  касается  событий  окружности,  заметим,  что  каждое  такое  обработанное  событие  определяет  вершину  $Vor(P)$.
	Напомним,  что  ложные  тревоги  удаляются  из  Q  еще  до  обработки.
	Они  создаются  и  удаляются  в  процессе  обработки другого --- настоящего --- события,  и  затрачиваемое  на  них  время  учитывается  во времени  обработки  этого  события.  
	Поэтому  количество  обрабатываемых  событий окружности  не  превосходит $2n - 5$.
	Оценки  времени  и  размера  памяти  доказаны.
\end{proof}