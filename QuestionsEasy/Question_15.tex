\section{Многочлены. Метод Горнера. Умножение Карацубы.}

В билете будем рассматривать многочлен $p_n(x) = a_0 + a_1x + a_2x^2 + \dots + a_n x^n$ с действительными коэффициентами и $a_i, x \in \mathbb{R}$

\textbf{Задача:} вычислить значение многочлена от конкретного $x$ за наименьшее количество сложных действий (то есть умножений чисел)

\textbf{Наивный способ:} посчитать в лоб (займёт $O(n^2)$ умножений)

\textbf{Схема Горнера}.
Постараемся перегруппировать действия, чтобы не было долгих возведений в $n$-ю степень.

Тогда многочлен $p(x)$ представим в виде:
$$
	p(x) = a_0 + x(a_1 + x(a_2 + x(a_3 + \dots x(a_{n-1} + a_n x))\dots))
$$
В данном способе потребуется всего $O(n)$ умножений.

\textbf{Задача} Быстрое умножение чисел на компьютере

\textbf{Наивная реализация} --- обычный метод умножения чисел в столбик.
В нём каждое число рекуррентно делится на две части и каждая часть числа умножается на каждую часть другого числа.
В подсчёте сложности получается такое рекуррентное соотношение
$$T(n) = 4T\left(\frac{n}{2}\right) + O(n)$$
Его решение $T(n) = O(n^2)$

\textbf{Метод Карацубы}
Представим наши числа в виде $n_1 = ax + b$, $n_2 = cx + d$ (данная операция осуществляется с помощью битового сдвига) и тогда задача сводится к вычислению коэффициентов многочлена $(ax+b)(cx+d)$
(данное соотношение можно заметить, раскрыв скобки в выражении $(a + b)(c + d)$)
\begin{align*}
	(ax+b)(cx+d) &= acx^2 + bcx + axd + bd^2 =  \\
	&= acx^2 + ((a + b)(c + d) - ac - bd)x + bd
\end{align*}
	
В данном способе необходимо выполнить лишь три сложных действия на каждом шаге: умножения $(a + b)(c + d)$, $ac$ и $bd$


Таким образом получается рекуррентное соотношение:
$$T(n) = 3T\left(\frac{n}{2}\right) + O(n)$$
Его решение $T(n) = O\left(n^{\log_2 3}\right)$, что быстрее наивного умножения в столбик.