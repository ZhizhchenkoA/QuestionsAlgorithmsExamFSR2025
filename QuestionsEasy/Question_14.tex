\section{kD-деревья. Окто- и квадро-деревья.}

\subsection*{KD-деревья}
\textbf{KD-деревья} --- статическая структура данных для хранения точек в $k$-мерном пространстве. Позволяет отвечать на запрос, какие точки лежат в данном прямоугольнике.

\subsubsection*{Построение дерева:}
\begin{enumerate}
	\item Разобъём все точки вертикальной прямой так, чтобы слева (нестрого) и справа (строго) от неё было примерно поровну точек (для этого посчитаем медиану первых координат). Получим подмножества для левого и правого ребёнка.
	\item Аналогично строим для этих подмножеств деревья, но разбивать будем уже не вертикальной, а горизонтальной прямой. И т.д.
	\item Повторяем 1-2 для каждой половины. (Если $k>2$, то заменим прямые на гиперплоскость)?
\end{enumerate}
\textbf{Сложность построения:} $T(n)=O(n)+2T(n/2) \Rightarrow T(n)=O(n \log n)$ - по мастер-теореме.
В многомерном случае: $O(n \log^{d-1}n)$, где $d$ - размерность.

\subsubsection*{Поиск:}
Ответить на вопрос, находится ли точка по запросу (количество точек в прямоугольнике).
Пусть нам поступил какой-то прямоугольник $RR$. Нужно вернуть все точки, которые в нём лежат. Будем это делать рекурсивно, получая на вход корень дерева и сам прямоугольник $RR$. Обозначим область, соответствующую вершине $v$, как $region(v)$. Она будет прямоугольником, одна или более границ которого могут быть на бесконечности. $region(v)$ можно явно хранить в узлах, записав при построении, или же считать при рекурсивном спуске. Если корень дерева является листом, то просто проверяем одну точку и при необходимости репортим её. Если нет, то смотрим, пересекают ли регионы детей прямоугольник $RR$. Если да, то запускаемся рекурсивно от такого ребёнка. При этом, если регион полностью содержится в $RR$, то можно репортить сразу все точки из него. Тем самым мы, очевидно, вернём все нужные точки и только их.

\textbf{Сложность:} $O(\log^2 k)$ в плоском случае в 2D.
В многомерном случае: $O(k + \log^d n)$, где $k$ - размер ответа, $d$ - размерность.

\subsection*{Окто- и квадро-деревья}
\textbf{Окто-дерево} - в $3D$. У узла $8$ потомков.
\textbf{Квадро-дерево} - в $2D$. У узла $4$ потомка.

\subsubsection*{Построение в 2D (3D аналогично):}
\begin{enumerate}
	\item Возьмём квадрат, в котором находятся все точки.
	\item Если всего одна точка, то дерево состоит из 1 листа с этой точкой. Иначе корнем делаем вершину, которая соответствует квадрату.
	\item Делим квадрат на 4 равные части и вызываем алгоритм для каждой части.
\end{enumerate}
\textbf{Глубина квадро-дерева} для множества точек $P$: $h \sim \log_2 \frac{L}{\varepsilon} + C$, где $\varepsilon$ - минимальное расстояние между точками из $P$, $L$ - сторона исходного квадрата.
\textbf{Квадро-дерево содержит} $O(n)$ вершин, где $n$ - всего точек.
\textbf{Построение:} $O(n \cdot \text{depth})$.
